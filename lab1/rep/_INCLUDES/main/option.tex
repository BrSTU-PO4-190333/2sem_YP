\paragraph{Условие:} \hspace{0pt}



Дана функция:

$$
f(x) = a \cdot x^2
+ b \cdot x^3
- c \cdot x
- d \cdot e^x
+ f_1 \cdot \cos{(f_2 \cdot x)}
+ g_1 \cdot \sin{(g_2 \cdot x)}
+ h_1 \cdot \cos^2{(x)}
+ h_2 \cdot \sin^2{(x)}
$$

где $a = $ кол-во гласных букв вашей фамилии.

$b = 0$ - если количество букв имени четное, и $ = 1$ - если кол-во букв нечетное.

$c = $ количество букв фамилии / количество букв имени ($ 0.01 $).

$d = 1$ - если количество букв имени четное, и $ = 0$ - если кол-во букв нечетное.

$f_1 = 0$ - если количество букв фамилии четное, и $ = 1$ - если кол-во букв нечетное.

$f_2 = $ количество букв в фамилии.

$g_1 = 1$ - если количество букв фамилии четное, и $ = 0$ - если кол-во букв нечетное.

$g_2 = $ количество букв в имени.

$h_1 = (f_2 + a) \cdot b$. $h_2 = (g_2 + a) \cdot d$.

Написать программу табуляции функции в заданном диапазоне. Диапазон вводится пользователем. Вывод сделать в интервальном отображении, например:

\begin{lstlisting}[language=empty,]
#   x1      x2      f1      f2
0   0.0     0.1     1.22    1.24
1   0.1     0.2     1.24    1.255
2   0.2     0.3     1.255   1.39
...
9   0.9     1.0     2.93    3.1415

Примечание:
# - номер интервала,
x1 - левая граница интервала,
х2 - правая граница интервала,
f1 - значение функции в левой границе интервала,
f2 - значение функции в правой границе интервала.
\end{lstlisting}

С помощью вашей программы определить:\
\begin{enumerate}
    \item Определить $x_{max}$ при котором $f(x)$ принимает максимальное значение.
    \item Получить приближенное решение уравнения $f(x) = 0$ (если такое существует, и ближайшее к $0$, если корней много).
\end{enumerate}



\newpage



\paragraph{Реализация:} \hspace{0pt}




\lstinputlisting[language=C++, name=main.h,]
{../../src/langC/src/main.h}

\lstinputlisting[language=C++, name=main.c,]
{../../src/langC/src/main.c}




\lstinputlisting[language=C++, name=lab.h,]
{../../src/langC/src/lab/lab.h}

\lstinputlisting[language=C++, name=lab.c,]
{../../src/langC/src/lab/lab.c}



\newpage



\lstinputlisting[language=C++, name=f.h,]
{../../src/langC/src/f/f.h}

\lstinputlisting[language=C++, name=f.c,]
{../../src/langC/src/f/f.c}



\lstinputlisting[language=C++, name=task0.h,]
{../../src/langC/src/task0/task0.h}

\lstinputlisting[language=C++, name=task0.c,]
{../../src/langC/src/task0/task0.c}




\lstinputlisting[language=C++, name=task1.h,]
{../../src/langC/src/task1/task1.h}

\lstinputlisting[language=C++, name=task1.c,]
{../../src/langC/src/task1/task1.c}



\newpage



\lstinputlisting[language=C++, name=task2.h,]
{../../src/langC/src/task2/task2.h}

\lstinputlisting[language=C++, name=task2.c,]
{../../src/langC/src/task2/task2.c}



\newpage



\paragraph{Результат программы:} \hspace{0pt}



\begin{lstlisting}[language=Out,]
Your name (Pavel): Pavel
Your surname (Galanin): Galanin
Left border (-3.9): -3.9
Right border (-3.7): -3.7
Step (0.01): 0.01
            #	          x1	          x2	          f1	          f2
            1	      -3.900	      -3.890	      -2.959	      -2.652
            2	      -3.890	      -3.880	      -2.652	      -2.347
            3	      -3.880	      -3.870	      -2.347	      -2.043
            4	      -3.870	      -3.860	      -2.043	      -1.742
            5	      -3.860	      -3.850	      -1.742	      -1.442
            6	      -3.850	      -3.840	      -1.442	      -1.145
            7	      -3.840	      -3.830	      -1.145	      -0.849
            8	      -3.830	      -3.820	      -0.849	      -0.556
            9	      -3.820	      -3.810	      -0.556	      -0.265
            10	      -3.810	      -3.800	      -0.265	       0.024
            11	      -3.800	      -3.790	       0.024	       0.311
            12	      -3.790	      -3.780	       0.311	       0.596
            13	      -3.780	      -3.770	       0.596	       0.878
            14	      -3.770	      -3.760	       0.878	       1.158
            15	      -3.760	      -3.750	       1.158	       1.436
            16	      -3.750	      -3.740	       1.436	       1.712
            17	      -3.740	      -3.730	       1.712	       1.985
            18	      -3.730	      -3.720	       1.985	       2.256
            19	      -3.720	      -3.710	       2.256	       2.524
            20	      -3.710	      -3.700	       2.524	       2.790
            21	      -3.700	      -3.690	       2.790	       3.053

Max value
f(-3.700000) = 2.789737

Approximation (1): 1
f(x) ~= 0:
f(-3.830000) = -0.849350
f(-3.820000) = -0.556006
f(-3.810000) = -0.264779
f(-3.800000) = 0.024299
f(-3.790000) = 0.311194
f(-3.780000) = 0.595875
f(-3.770000) = 0.878307
\end{lstlisting}