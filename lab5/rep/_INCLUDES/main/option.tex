\paragraph{Условие:} \hspace{0pt}

\begin{enumerate}
    \item Вспомнить: указатели, ссылки; структуры данных (записи); динамические структуры: стек, дек; чтение данных из файла; функции: malloc, sizeof, free; new, delete.
    \item Разобраться с алгоритмом кодирования Шеннона-Фано и Хаффмана.
    \item Разобраться с принципами построения деревьев. Разработать подход построения бинарного дерева, который реализует соответствующий (вашему варианту) алгоритм кодирования.
    \item Написать программу:
    \begin{enumerate}
        \item Считать текст из файла. Это исходный текст, на основании которого будет происходить кодирование.
        \item Составить статистику по символам, встречающимся в тексте, в отсортированном виде: <Символ> <Частота> ... ...
        \item Реализовать необходимые типы/структуры для организации дерева. Разработать функции для работы с деревьями. Протестировать их работу, прежде чем приступать к реализации алгоритма кодирования.
        \item Реализовать алгоритм построения дерева.
        \item Отобразить (сохранить в файл) таблицу кодов для символов исходного текста.
    \end{enumerate}
\end{enumerate}

\newpage


\paragraph{Реализация:} \hspace{0pt}

\lstinputlisting[language=C++, name=main.c,]
{../../src/langC/src/main.c}

\newpage


\paragraph{Результат программы:} \hspace{0pt}

\begin{lstlisting}[language=Out,]
= = = File = = =
Data Reduction
Data reduction techniques are used in order to obtain a new representation
of the data set that is much smaller in volume, but yet produces
the same (or almost the same) analytical results. The most common
reduction strategies are a) data cube aggregation, b) dimensionality reduction,
c) numerocity reduction, and d) discretization and concept hierarchy
generation [Barbara et al., 1997].
2.6 Data Discretization
As already mentioned, data discretization is a strategy for data reduction,
but with particular importance. Discretization methods are applied
both on numeric and categorical data. Specific methods include:
- Discretization methods for numeric data [Kerber, 1992]
- Binning
- Histogram Analysis
- Cluster Analysis
- Entropy-Based Discretization
- Segmentation by natural partitioning
- Discretization methods for categorical data [Han and Fu, 1994]
- Specification of a partial ordering of attributes explicitly at the
schema level by users or experts
Data Mining and Knowledge Discovery: A brief overview 21
- Specification of a portion of a hierarchy by explicit data grouping
- Specification of a set of attributes, but not of their partial ordering
�
= = = End file = = =

= = = Array = = =
| array[i] | counter  | symbal   | code             |
| -------- | -------- | -------- | ---------------- |
| 0        | 155      |          | __000            |
| 1        | 100      | t        | __001            |
| 2        | 98       | a        | __0100           |
| 3        | 94       | e        | __0101           |
| 4        | 92       | i        | __0110           |
| 5        | 69       | r        | __0111           |
| 6        | 67       | o        | __1000           |
| 7        | 64       | n        | __1001           |
| 8        | 46       | c        | __10100          |
| 9        | 45       | s        | __10101          |
| 10       | 38       | d        | __10110          |
| 11       | 28       | u        | __10111          |
| 12       | 27       | l        | __11000          |
| 13       | 23       | \n       | __110010         |
        | __110011         |
| 15       | 22       | m        | __110100         |
| 16       | 20       | h        | __110101         |
| 17       | 20       | p        | __110110         |
| 18       | 18       | g        | __110111         |
| 19       | 16       | y        | __111000         |
| 20       | 16       | f        | __111001         |
| 21       | 15       | b        | __1110100        |
| 22       | 11       | -        | __1110101        |
| 23       | 10       | ,        | __1110110        |
| 24       | 10       | D        | __1110111        |
| 25       | 7        | z        | __1111000        |
| 26       | 6        | .        | __11110010       |
| 27       | 6        | 9        | __11110011       |
| 28       | 5        | S        | __11110100       |
| 29       | 5        | )        | __11110101       |
| 30       | 5        | v        | __11110110       |
| 31       | 4        | w        | __11110111       |
| 32       | 4        | 1        | __11111000       |
| 33       | 4        | A        | __111110010      |
| 34       | 3        | [        | __111110011      |
| 35       | 3        | ]        | __11111010       |
| 36       | 3        | x        | __111110110      |
| 37       | 3        | 2        | __111110111      |
| 38       | 3        | B        | __111111000      |
| 39       | 2        | H        | __111111001      |
| 40       | 2        | :        | __111111010      |
| 41       | 2        | K        | __1111110110     |
| 42       | 1        | 6        | __1111110111     |
| 43       | 1        | M        | __1111111000     |
| 44       | 1        | T        | __1111111001     |
| 45       | 1        | �        | __1111111010     |
| 46       | 1        | 7        | __11111110110    |
| 47       | 1        | (        | __11111110111    |
| 48       | 1        | q        | __1111111100     |
| 49       | 1        | R        | __11111111010    |
| 50       | 1        | 4        | __11111111011    |
| 51       | 1        | F        | __1111111110     |
| 52       | 1        | E        | __11111111110    |
| 53       | 1        | C        | __111111111110   |
= = = End array = = =
\end{lstlisting}
