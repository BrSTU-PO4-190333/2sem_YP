\paragraph{Условие:} \hspace{0pt}

Ознакомиться с понятием градиента и методом градиентного спуска (Вспомнить дифференцирование функций)

Написать программу поиска максимумов и минимумов функции $F(x,y,z)$ методом градиентного поиска (с постоянным шагом).

Входные данные в программу: Пользователь 
\begin{itemize}
    \item вводит диапазон поиска
    \item вводит стартовую точку поиска (можно генерировать случайно в указанном диапазоне)
    \item выбирает направление поиска ($min, max$).
\end{itemize}

Выходные данные: Найденный экстремум - точка и соответствующее значение функции.

Вид функции и вариант:

$$F(x,y,z) = f_1*f_2 + f_3*f_4$$

\begin{center}
    \begin{tabular}{ | l | l | l | l | l | }
        \hline
        Вариант & $f_1$ & $f_2$ & $f_3$ & $f_4$ \\
        \hline
        $5$ & $x \cdot y$ & $z^2$ & $2 \cdot x^2 + 3$ & $y^2$ \\
        \hline
    \end{tabular}
\end{center}


\paragraph{Реализация:} \hspace{0pt}


\lstinputlisting[language=C++, name=main.h,]
{../../src/langC/src/main.h}

\lstinputlisting[language=C++, name=main.c,]
{../../src/langC/src/main.c}


\lstinputlisting[language=C++, name=lab.h,]
{../../src/langC/src/lab/lab.h}

\lstinputlisting[language=C++, name=lab.c,]
{../../src/langC/src/lab/lab.c}


\lstinputlisting[language=C++, name=gradXYZ.h,]
{../../src/langC/src/gradXYZ/gradXYZ.h}

\lstinputlisting[language=C++, name=gradXYZ.c,]
{../../src/langC/src/gradXYZ/gradXYZ.c}


\lstinputlisting[language=C++, name=option_5.h,]
{../../src/langC/src/option_5/option_5.h}

\lstinputlisting[language=C++, name=option_5.c,]
{../../src/langC/src/option_5/option_5.c}


\paragraph{Результат программы (поиск min'имума):} \hspace{0pt}

\begin{lstlisting}[language=Out,]
Шаг (0.001): 0.001
Приближение (0.001): 0.001 
Искать min - 1, max - 2 (1): 1
Левая граница X (-5): -5
Правая граница X (5): 5
Начальная точка X (1): 1
Левая граница Y (-5): -5
Правая граница Y (5): 5
Начальная точка Y (1): 1
Левая граница Z (-5): -5
Правая граница Z (5): 5
Начальная точка Z (1): 1

Min-ум функции:
f(
    x = 2.493378
    y = -1.521199
    z = 5.023123
) = -59.987482
\end{lstlisting}


\paragraph{Результат программы (поиск max'имума):} \hspace{0pt}

\begin{lstlisting}[language=Out,]
Шаг (0.001): 0.001
Приближение (0.001): 0.001
Искать min - 1, max - 2 (1): 2
Левая граница X (-5): -5
Правая граница X (5): 5
Начальная точка X (1): 1
Левая граница Y (-5): -5
Правая граница Y (5): 5
Начальная точка Y (1): 1
Левая граница Z (-5): -5
Правая граница Z (5): 5
Начальная точка Z (1): 1

Max-ум функции:
f(
    x = 4.471862
    y = 5.133971
    z = 1.874665
) = 1213.934496    
\end{lstlisting}